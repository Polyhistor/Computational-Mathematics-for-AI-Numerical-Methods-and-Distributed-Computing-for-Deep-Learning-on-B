% (NB Remove when finalising)
% Suppress \hbox warnings
\sloppy

Deep learning has emerged as a transformative technology, providing state-of-the-art solutions for a wide range of big data applications.
However, as the complexity of these models grows and data volumes continue to increase, there is a significant need to understand and optimize the computational methods underpinning these systems. Numerical methods and distributed computing play pivotal roles in addressing the computational challenges associated with training and deploying deep learning models on large-scale datasets. Recent advancements in this field have led to various approaches for optimizing performance, scalability, and resource efficiency.

One of the key challenges in deep learning is efficiently handling the vast amounts of data and computational requirements involved in training deep learning models.
Numerical methods such as optimization algorithms are fundamental for training these models, particularly in ensuring convergence and minimizing loss functions effectively.
As highlighted by \citet{najafabadi2015deep}, the integration of deep learning techniques with big data analytics presents numerous challenges, particularly in terms of computational efficiency and scalability. This necessitates a deeper exploration of computational mathematics to improve model training and inference.

In addition to numerical methods, distributed computing techniques have become increasingly crucial in the context of big data and deep learning. \citet{yan2023computational} outlines the theoretical foundations and practical implementations of computational methods for deep learning, emphasizing the importance of distributed frameworks.

Techniques such as GPU acceleration, federated learning, and parallel processing are instrumental in scaling deep learning models to meet the demands of large-scale data processing \citep{dehghani2023distributed}. These distributed computing approaches enable more efficient training by distributing workloads across multiple nodes or devices, thus reducing training time and improving scalability.

Overall, the intersection of numerical methods and distributed computing forms the backbone of scalable deep learning systems for big data applications.
By synthesizing knowledge from both domains, it is possible to create more efficient deep learning models capable of processing large datasets with reduced computational overhead.
